\documentclass[a4paper,11pt]{article}
\usepackage[T1]{fontenc}
\usepackage[utf8]{inputenc}
\usepackage{lmodern}
\usepackage[francais]{babel}
\usepackage{listings}
% pour definir des couleurs
\usepackage{xcolor}
\usepackage{amsmath}


% code color
\definecolor{ligthyellow}{RGB}{250,247,220}
\definecolor{darkblue}{RGB}{5,10,85}
\definecolor{ligthblue}{RGB}{1,147,128}
\definecolor{darkgreen}{RGB}{8,120,51}
\definecolor{darkred}{RGB}{160,0,0}

\lstset{
    language=Scilab,
    captionpos=b,
    extendedchars=true,
    frame=lines,
    numbers=left,
    numberstyle=\tiny,
    numbersep=5pt,
    keepspaces=true,
    breaklines=true,
    showspaces=false,
    showstringspaces=false,
    breakatwhitespace=false,
    stepnumber=1,
    showtabs=false,
    tabsize=3,
    basicstyle=\small\ttfamily,
    backgroundcolor=\color{ligthyellow},
    keywordstyle=\color{ligthblue},
    morekeywords={include, printf, uchar},
    identifierstyle=\color{darkblue},
    commentstyle=\color{darkgreen},
    stringstyle=\color{darkred},
}


\title{Rapport TP objet déformable}
\author{Elliot Vanegue et Gaëtan Deflandre}

\begin{document}

\maketitle
%\tableofcontents

\section{Introduction}
Lors de ces TPs, nous étudions l'application de forces sur le maillage d'une
forme. L'application des forces se fait alors sur chaque point du maillage de la forme.
Le but est de simuler un comportement cohérent d'un objet en mouvement. Pour cela, nous 
commençons par simuler la gravité sur les points d'un maillage et nous ajoutons
ensuite des ressorts sur ces points afin d'obtenir un objet déformable. Ensuite, nous 
gérons les contacts avec l'objet ainsi que la réponse à la collision.  

\section{Modèle masse-ressort}
\subsection{Calcul de la masse}
Dans un premier temps, nous appliquons la gravité sur chacun des points.
Pour cela, nous créons un vecteur accélération de la forme [0;-g], g étant la constante
de gravité. Nous mettons cette constante en négatif pour que la force aille vers le bas
du graphique pour simuler la gravité terrestre. Ensuite, nous calculons la vitesse comme ceci :
\begin{equation}
vitesse(t) = vitesse(t-1) + acceleration
\end{equation}
L'accélération reste constante dans le temps, mais la vitesse, elle, augmente jusqu'à un certain
point (jusqu'à ce que la force de gravitation soit égale aux frottements). Et enfin, nous calculons la position grâce au calcul suivant : 
\begin{equation}
position(t) = position(t-1) + vitesse(t)
\end{equation}

Voici le code que nous réalisons pour ce calcul.
\begin{lstlisting}[caption=Code du Calcul de la gravité]
for i=1:numNoeuds
  //calcule chute libre
  acceleration = [0;(-g * dt)];
  if(time-dt == 0)
    continue;
  end
  vitesse = oldVitesse + acceleration;
  X_t([2*i-1 2*i]) = X_t([2*i-1 2*i]) + vitesse;
end
    
oldVitesse = vitesse;
\end{lstlisting}

\subsection{Calcul du ressort}
Pour cette partie, nous n'avons pas réussi à calculer la force du ressort, nous avons calculé la longueur 
à l'instant t (L(s)) et la force F comme étant :
\begin{equation}
F = k*(L(s)-L0(s))
\end{equation}
avec L0(s) la longueur initiale.
Nous avons compris qu'il fallait multiplier notre calcul par la normale du point afin que cette force
s'applique dans la direction de cette normale. Le calcul de la normale se fait comme ceci:
\begin{equation}
n = (p1 - p2) * (\frac{1}{L(s)})
\end{equation}
On peut remarquer que lorsque nous augmentons trop la valeur dt les points partent n'importe où
et l'objet ne ressemble plus du tout à un rectangle. En revanche, lorsque nous diminuons dt,
les mouvements des points du maillage sont moins importants.

\section{Intégration implicite}
L'intégration implicite permet d'obtenir une simulation plus stable en limitant l'erreur d'arrondi
des résultats. L'application de l'intégration implicite va transformer nos calculs précédents de 
la manière suivante :
\begin{align}
  acceleration(t) &= dv / dt\\
  position(t) &= position(t-1) + (vitesse(t-1) + dv) * dt\\
  dv &= vitesse(t) - vitesse(t-1) 
\end{align}
Les calculs pour obtenir la force du ressort entre chaque point nous est toujours utile. C'est pourquoi
nous mettons ces calculs dans une fonction avec comme paramètre la position des points calculés à l'instant t.
Cette position se calcule de la manière suivante :
\begin{equation}
  position(t-1) + (vitesse(t-1) + dv)*dt
\end{equation}
Ce qui nous permet de réaliser la fonction d'intégration implicite de la manière suivante :
\begin{equation}
  Fnl(dv) = m * dv/dt - m*G - Fk(position(t-1) + (vitesse(t-1) + dv)*dt)
\end{equation}

\begin{lstlisting}[caption=Fonction permettant de réaliser l'intégration implicite]
function [value] = Fnl(dV)
    

    numNoeuds = size(noeuds,2);  
    F_t = zeros(2*numNoeuds,1);
    G = size(2*numNoeuds,1);
    for i=1:numNoeuds
        G(2*i) = -g;
    end
    for i=1:5
        dV(2*i) = 0;
        dV(2*i-1) = 0;
    end
    
    // reecrire les equations de newton en implicite
    F_t = m * dV/dt - m*G - Fk(_MYDATA_.X_t + (_MYDATA_.V_t + dV) * dt);
    
    for i=1:5
        F_t(2*i) = 0;
        F_t(2*i-1) = 0;
    end
    
    value = F_t;
    
endfunction

function [value] = Fk(position)
    F_t = zeros(2*numNoeuds,1);
    L=[];
    for s=1:numSegments,
        i1= segments(1,s);
        i2= segments(2,s);
        // longueur actuelle
        L(s) = norm( position([2*i1-1 2*i1]) - position([2*i2-1 2*i2]) )
        // normale actuelle
        n = (position([2*i1-1 2*i1]) - position([2*i2-1 2*i2]))*(1.0/L(s));
   
        // force du ressort
        F = n*(k*(L(s)-L0(s)));
        
        // ajout de la force dans le vecteur
        F_t([2*i1-1 2*i1]) = F_t([2*i1-1 2*i1])  - F;
        F_t([2*i2-1 2*i2]) = F_t([2*i2-1 2*i2])  + F;
    end
    
    value = F_t;

endfunction
\end{lstlisting}

\section{Gestion des contacts}
\subsection{Méthode de pénalité}
\subsubsection{Pénalité explicite (cas rigide)}
Nous simulons, dans un premier temps, la chute d'un maillage rigide. Il y aura alors contact avec un plan.
Nous calculons d'abord la masse complète de l'objet (mG), ici un disque. Nous avons à notre disposition la 
masse volumique (mv) égale à 0,027 kg/$cm^3$ et l'épaisseur du disque égale à 0.5 cm.
\begin{equation}
  mG = mass\_vol * (\Pi*5*5*epaisseur)
\end{equation}
Il faut alors appliquer cette masse à l’accélération de l'objet afin que celui-ci soit mis en mouvement.
\begin{equation}
  aG = \frac{[0;-981]*mG + FC}{mG}
\end{equation}
FC est la force de contact que nous calculerons par la suite.
Etant donné que le plan avec lequel l'objet va rentrer en collision est en y=-2, il suffit de vérifier
que l'un des points est en dessous de ce plan, donc de -2, pour détecter la collision. Pour répondre à
cette collision et permettre à l'objet d'être repoussé par ce plan, il faut appliquer une force de pénalité
sur l'axe des y. Cette force de pénalité va alors faire rebondir l'objet.
\begin{equation}
  FC(2) = FC(2) - k * (2 + noeuds_deplaces(2,i))
\end{equation}
Nous pouvons ensuite ajouter une composante d'amortissement à cette force, ce qui va permettre d'atténuer
le rebondissement de l'objet. Plutôt que de faire un grand saut comme c'était le cas précédemment, il fera 
une succession de petits sauts de plus en plus petits jusqu'à ce qu'il soit stable sur le plan.
\begin{equation}
  FC(2) = FC(2) - k * (2 + noeuds_deplaces(2,i)) - b * vG(2)
\end{equation}
\begin{lstlisting}[caption=Calcul de la pénalité explicite]
  mG = mass_vol * (%pi*5*5*epaisseur);
  for time=0:dt:T,
  
    // calcul de la dynamique d'un corps rigide  (integration explicite)
    aG = ([0;-981]*mG + FC) /mG;   
    vG = vG + aG*dt
    pG = pG + vG*dt;
    FC=[0;0];
    
    // calculer le deplacement du maillage en fonction du deplacement du centre de gravite
    noeuds_deplaces = noeuds;
    for i=1:numNoeuds,
      // deplacement selon x
      noeuds_deplaces(1,i) = pG(1) + noeuds(1,i);
      // deplacement selon y
      noeuds_deplaces(2,i) = pG(2) + noeuds(2,i);  
      
      //detection de collision
      if noeuds_deplaces(2,i) < -2,
          FC(2) = FC(2) - k * (2 + noeuds_deplaces(2,i)) - b * vG(2);
      end
    end 
  end
\end{lstlisting}
\subsubsection{Pénalité implicite (cas déformable)}
Dans le cas d'un objet déformable, nous calculons la distance de chaque point avec le plan, nous 
vérifions si cette distance est négative. Si cette distance est négative, on applique une force inverse
sur chaque point du maillage.
\begin{lstlisting}
for i=1:numNoeuds
  distance = X_t(2*i) - (-1);
  if(distance <= 0),
      F_t(2*i) = F_t(2*i) + -k * abs(distance);            
  end 
end
\end{lstlisting}


% \subsection{Multiplicateur de Lagrange}
% \subsubsection{Matrice tangente d'un système masse-ressort}
% \begin{lstlisting}[caption=Calcul de matrice tangente du système]
%   function [F,K] = computeSpring(P1,P2, LO, k)
% 	// longueur actuelle
% 	L = norm( P2 - P1 )
% 	// normale actuelle
% 	n = (P2 - P1)*(1.0/L);
% 	// force du ressort
% 	forceIntensity = k*(L-L0);
% 
% 	// force sur les noeuds
% 	F = [n* forceIntensity; -n* forceIntensity];
% 
% 	// tangent stiffness 
%     tgt = forceIntensity / L;
%     kr = n*n'*(k-tgt) + [1 0;0 1] * tgt;
%     K = [-kr kr; kr -kr]
% 
% endfunction
% \end{lstlisting}
% \subsubsection{Solveur de LCP}
% \begin{lstlisting}[caption=Calcul de l'intensité des forces de contact]
%   function [lambda, delta] = GaussSeidel(W,dfree, maxIteration)
%     
%     num=size(W,1);
%     lambda=zeros(num,1);
%     delta = zeros(num,1);
%     // ecrire l'algorithme
%     
%     for i=1 :maxIteration
%         for c=1:num
%             lambda(c)=0;
%             d=W(c,:) * lambda + dfree(c)
%             if (d<0)
%                 lambda(c) = -d/W(c,c);
%             end
%         end
%     end
%     
% endfunction
% \end{lstlisting}
\section{Conclusion}
Nous avons vu lors de ces TPs, comment appliquer des forces sur un objet rigide, puis comment
étendre ces forces à l'ensemble d'un maillage afin d'obtenir un objet déformable. L'application
de ressort entre chaque point du maillage de l'objet permet d'avoir un bon rendu de corps déformable.
\end{document}
