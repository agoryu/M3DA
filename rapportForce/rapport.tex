\documentclass[a4paper,11pt]{article}
\usepackage[T1]{fontenc}
\usepackage[utf8]{inputenc}
\usepackage{lmodern}
\usepackage[francais]{babel}
\usepackage{listings}
% pour definir des couleurs
\usepackage{xcolor}
\usepackage{amsmath}


% code color
\definecolor{ligthyellow}{RGB}{250,247,220}
\definecolor{darkblue}{RGB}{5,10,85}
\definecolor{ligthblue}{RGB}{1,147,128}
\definecolor{darkgreen}{RGB}{8,120,51}
\definecolor{darkred}{RGB}{160,0,0}

\lstset{
    language=C++,
    captionpos=b,
    extendedchars=true,
    frame=lines,
    numbers=left,
    numberstyle=\tiny,
    numbersep=5pt,
    keepspaces=true,
    breaklines=true,
    showspaces=false,
    showstringspaces=false,
    breakatwhitespace=false,
    stepnumber=1,
    showtabs=false,
    tabsize=3,
    basicstyle=\small\ttfamily,
    backgroundcolor=\color{ligthyellow},
    keywordstyle=\color{ligthblue},
    morekeywords={include, printf, uchar},
    identifierstyle=\color{darkblue},
    commentstyle=\color{darkgreen},
    stringstyle=\color{darkred},
}


\title{Rapport TP objet déformable}
\author{Elliot Vanegue}

\begin{document}

\maketitle
\tableofcontents

\section{Introduction}
Lors de ces TP, nous avons travaillé sur l'application de différentes forces 
sur un objet représenté sous forme d'un maillage. L'application des forces 
se fait alors sur chaque point du maillage de la forme.

\section{Calcule de la masse}
Dans un premier temps, nous allons appliquer la gravité sur chacun des points.
Pour cela, j'ai créé un vecteur accélération de la forme [0;-g], g étant la constante
de gravité. J'ai mis cette constante en négatif pour la force aille vers le bas
du graphique pour simuler la graviter terrestre. Ensuite, j'ai calculé la vitesse comme ce-ci :
\begin{equation}
vitesse(t) = vitesse(t-1) + acceleration
\end{equation}
Et enfin, j'ai calculé la position grâce au calcul suivant : 
\begin{equation}
position(t) = position(t-1) + vitesse(t)
\end{equation}

Voici le code que j'ai réalisé pour ce calcul.
\begin{lstlisting}[caption=Code du Calcul de la gravité]
for i=1:numNoeuds
  //calcule chute libre
  acceleration = [0;(-g * dt)];
  if(time-dt == 0)
    continue;
  end
  vitesse = oldVitesse + acceleration;
  X_t([2*i-1 2*i]) = X_t([2*i-1 2*i]) + vitesse;
end
    
oldVitesse = vitesse;
\end{lstlisting}

\section{Calcul du ressort}
Pour cette partie, je n'ai pas réussi à calculer la force du ressort, j'ai simplement réussi
à calculer la longueur à l'instant t et j'ai calculé la force F comme étant :
\begin{equation}
F = k*(L(s)-L0(s))
\end{equation}
J'ai compris qu'il fallait multiplier mon calcul par la normal du point afin que cette force
s'applique dans la direction de cette normal. Le calcul de la normal se fait comme ce-ci:
\begin{equation}
n = (p1 - p2) * (\frac{1}{L(s)})
\end{equation}
On peut remarquer que lorsque nous augmentons trop la valeur dt les points parte n'importe où
et l'objet ne ressemble plus du tout à un rectangle. En revanche, lorsque nous diminuons dt,
les mouvements des points du maillage sont moins important.

\section{Intégration implicite}
L'intégration implicite permet d'obtenir une simulation plus stable en limitant l'erreur d'arrondi
des résultats. L'application de l'intégration implicite va transformer nos calculs précédents de 
la manière suivante :
\begin{align}
  acceleration(t) &= dv / dt\\
  position(t) &= position(t-1) + (vitesse(t-1) + dv) * dt\\
  dv &= vitesse(t) - vitesse(t-1) 
\end{align}
Dans un premier temps, j'ai mis dans une fonction l'ensemble des calculs réalisés
dans les parties précédentes en mettant comme agrument la position du point calculer à l'instant t.
Cette position correspond à :
\begin{equation}
  position(t-1) + (vitesse(t-1) + dv)*dt
\end{equation}
Ce qui nous permet de réaliser la fonction d'intégration implicite de la manière suivante :
\begin{equation}
  Fnl(dv) = m * dv/dt - m*G - Fk(position(t-1) + (vitesse(t-1) + dv)*dt)
\end{equation}
\begin{lstlisting}[caption=Fonction permettant de réaliser l'intégration implicite]
function [value] = Fnl(dV)
    

    numNoeuds = size(noeuds,2);  
    F_t = zeros(2*numNoeuds,1);
    G = size(2*numNoeuds,1);
    for i=1:numNoeuds
        G(2*i) = -g;
    end
    for i=1:5
        dV(2*i) = 0;
        dV(2*i-1) = 0;
    end
    
    // reecrire les equations de newton en implicite
    F_t = m * dV/dt - m*G - Fk(_MYDATA_.X_t + (_MYDATA_.V_t + dV) * dt);
    
    for i=1:5
        F_t(2*i) = 0;
        F_t(2*i-1) = 0;
    end
    
    value = F_t;
    
endfunction

function [value] = Fk(position)
    F_t = zeros(2*numNoeuds,1);
    L=[];
    for s=1:numSegments,
        i1= segments(1,s);
        i2= segments(2,s);
        // longueur actuelle
        L(s) = norm( position([2*i1-1 2*i1]) - position([2*i2-1 2*i2]) )
        // normale actuelle
        n = (position([2*i1-1 2*i1]) - position([2*i2-1 2*i2]))*(1.0/L(s));
   
        // force du ressort
        F = n*(k*(L(s)-L0(s)));
        
        // ajout de la force dans le vecteur
        F_t([2*i1-1 2*i1]) = F_t([2*i1-1 2*i1])  - F;
        F_t([2*i2-1 2*i2]) = F_t([2*i2-1 2*i2])  + F;
    end
    
    value = F_t;

endfunction
\end{lstlisting}

\section{Conclusion}
\end{document}
